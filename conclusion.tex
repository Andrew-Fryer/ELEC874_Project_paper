\section{Conclusions}\label{sec:conclusion}

As mentioned in section \ref{sec:cedar_flaw}, there is at least one significant difference between the genuine and forge images in the CEDAR dataset.
Since CNNs are very powerful and should be able to pick up on this difference very easily, the CEDAR dataset is not a good test of an architecture's ability to understand or verfiy handwritten signatures.

SigNet was more resistent to the attack than expected.
Training for a longer time improved its performance.
Goodfellow says that adversarial regularizers complement dropout, but it seems that dropout may have done a good job.

Was this because of a flaw in the dataset???
I speculate that SigNet is learning info about the cameras/lighting conditions used for each person/or set up maybe?
...my thinking is that the pertebation looked like background noise, the genuine had all zeros in the background, and the forge had non-zero small values in the background...
This qualitatively shows that SigNet's [desicion] is largely influenced by [].

swapping out networks shows/disproves that a black box attack is nearly as effective as a white-box attack.

It seems that larger latent vectors are slightly better suited to prediction and slightly worse suited to resistence against adversarial attacks.
More tests are needed to collect statistically significant data.

The thresholding function does/does not matter.

Talk about losses?

