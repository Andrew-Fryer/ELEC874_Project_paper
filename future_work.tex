\section{Future Work}\label{sec:future_work}

The latent vector size expirement should be repeated many times to try to get statistically significant results...

Implement triplet loss on a SigNet-like network.

Work has already been done in improving network's by training them an adversarial examples created using FGSM and similar methods.

There was one area of [large] interest that [I] didn't have time to investigate.
LeCun assumes that the the cluster of the images of one person's face in the latent space is Gaussian.
We can easily compute the probability that the Gaussian distribution produces a point.
My understanding is:
Then, he/they average the probabilities for the impostor images for a subject to estimate p\_impostor (which is just a term that sort of scales the p value of an image to the genuine cluster so that ...)
This is just comparing the p value (probability that the Gaussian distribution for the genuines produced the point) to the averge p value of the impostors.
(If I understand this correctly,) This doesn't make sense!
Even if the mapping to embedded space seperates people perfectly this strategy will let something half of the impostors go because statistically something like half of the impostor's p value will be less than the average of all impostors' p values.
(not exactly half because it depends on what the distribution actually looks like...)
Instead, we could compute a p\_impostor value if we assume that the impostors also form a Gaussian distribution (which is not probable given that the model is trying to create a margin, but may result in good accuracy...)

In our one-shot (only 1 genuine image) system, we can't compute a Gaussian distribution for each person.
Instead, we could make the assumption that the distribution for each person looks the same for its genuines and also for its impostors.
Then, we can compute a distrubution for the genuines.
We could the approach that LeCun does here, (and say that we just sort of scale p using the average impostor p).
Or, we can assume that the impostor signatures also form a gaussian distribution.
Then, we can do the following:
\begin{flalign*}
is genuine & = P_g > P_i&\\
        & = Bayes&\\
        & = Multivariate Gaussian&\\
        & = take ln&\\
        & = simplify&\\
        & = -x^T * \sigma_g^-1 * x > -x^T * \sigma_i^-1 * x + 2 * (ln(det(\sigma)^-1/2) - ln(det(\sigma)^-1/2))&\\
\end{flalign*}
where g is for genuine and i is for impostor
where < is the computer science ``less than'' comparison operator rather than denoting an inequality.
\begin{eqnarray}
f(y_{i}|\mu,\Sigma) & =\nonumber \\
    & = & f((y_{i1},y_{i2},...,y_{in})'\,|\,\mu=(\mu_{1},...,\mu_{n})',\Sigma=\left[\begin{array}{ccc}
\sigma_{1}^{2} &  & \sigma_{1n}\\
    & \ddots\\
\sigma_{n1} &  & \sigma_{n}^{2}
\end{array}\right])\nonumber \\
    &  & =\frac{1}{\sqrt{(2\pi)^{n}|\Sigma|}}exp(-\frac{1}{2}(y_{i}-\mu)'\Sigma^{-1}(y_{i}-\mu))
\end{eqnarray}

If I didn't get to implementing a cool threshold I should talk about that here...
TODO: talk about Gaussian threshold idea
