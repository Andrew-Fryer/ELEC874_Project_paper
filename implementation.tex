\section{Implementation}\label{sec:implementation}


\subsection{General Strategy}
% todo: move related work section here and integrate

Take Signet.
Attack it using a technique similar to FGSM.
[Try to improve SigNet's resilience to the attack.]


\subsection{Procedure and Workarounds}

For implementing the SigNet, 3 resources were used.
The SigNet paper was consulted, which describes the model completely\cite{sig_net}.
The Keras code presumably by one of the paper authors was used for clarification\cite{GitHub_sounakdey}.
Lastly, an existing PyTorch implementation of SigNet was used as a starting place\cite{GitHub_signet_pytorch}.

The existing PyTorch implementation first restructured into a Jupyter Notebook.
Then, it was checked against the specification of the network in the paper.
The existing training process did not implement data normalization, so this was implemented as per the paper.
The mean and standard deviation of the images in the training dataset was computed (after spliting the images in the CEDAR dataset into training and validation images and pairing images).
For efficiency, all of the images in the dataset were then resized, transformed (inverted), and normalized according the paper's description and then saved back into png files.
[I didn't save the image data to disk as a .pt file (as a tensor) because I figured that png compression is probably better than the .pt compression since it is image-specific.]

% Training is taking way too long, so I'm going to try to get the following to work:
% -divide images once
% -resize and normalize images once
% -each time:
%     -load all images into memory as a `list` of tensors
%     -store a training and validation list of triples (ind_into_img_list, ind2, y/n)
% This way, we only read the images from disk at the beginning, so it should be a lot faster.

Only training on CEDAR due to time constraints
(Perhaps we'd get better accuracy if we trained on several datasets and validated on just CEDAR... did the SigNet paper do that?)


Unfortunately, it was later discovered that the last layer of the network was incorrect in the existing implementation and this went unnoticed.
The Signet implementation in this paper has [last layer output size is 256 instead of 128].
% would training go faster if this was fixed?
% I could also try to use the trained incorrect model to initialize the correct model...

The process of preparing the data is all the same.
While the paper uses several datasets, this paper only uses the CEDAR dataset due to time constraints [ref].
% todo: ref

Another small disrepency is in the initialization of model weights.
``We initialize the weights of the model according to the work
of Glorot and Bengio [20], and the biases equal to 0. We trained
the model using RMSprop for 20 epochs, using momentum rate
equal to 0.9, and mini batch size equal to 128.'' <- SigNet
% todo
% I'm not sure if the biases were correctly initialized to 0 or if the other weights were initialized with the same strategy...

Training was done on Google Colab (using the \_ level tier thing).
The training process took approximately 1 hour per 1000 [cases/datapoints].
The training process was changed to use a dataloader that stores all of the image data in computer memory instead of repeatedly accessing the information from disk in the form of PNG files.
This is feasible because the number of images is far fewer than the number of pairs of images (which are used as [cases/datapoints]).
Unfortunately, this did not seem to speed up the training process significantly.

I should try increasing the batch size!!!!

The model was trained for approximately \_ hours [until batch 19050] and then evaluated, yeilding an accuracy of 72\%.
Then, it was trained until the end of the first epoch (another \_ hours) and evaluated [validated?] again.
This time the accuracy was 94\%.

Although the paper trains the model for 20 epochs, training was stopped at 1 epoch due to time constraints and because 94\% accuracy seemed sufficient for the puposes of this paper.
(Unfortuantely weights from a trained SigNet model could not be obtained.)

I didn't do the weight initialization the same way (because I didn't do it the first time and I didn't have time to re-run stuff...)

\subsubsection{Loss}
Alpha and Beta in the contrastive loss are both 1, as in SigNet\cite{GitHub_sounakdey}.
It was decided not to implement triplet loss for SigNet because it would be substantial work and it seems that it would take far too long to train.


\subsection{Expirements}

[I] seek to validate the model as if it were used in a live production signature verification system, understand how effective a white-box attack would be, and understand how latent vector size would impact the accuracy and robustness of the system.

\subsubsection{Threshold Strategy}
The code in the repo presumably by the SigNet paper author computes accuracy given a thrsehold that is determined by taking the threshold that gives the best accuracy on the validation set.
IMHO, this is wrong because the threshold is information that is technically part of the classifier that is being leaked from the validation set.
See: \url{https://github.com/sounakdey/SigNet/blob/master/SigNet_v1.py#L84}
This approach is useful for understanding the potential of SigNet, but does not give an accuracy evaluation of the accuracy that SigNet would have in the real world because of this information leakage.


I should compute the divide as $p_same / (p_same + p_different)$ where each p is computed as the Gaussian distribution of the distances on the training data...

!!! I don't think that SigNet implements the decision (genuine vs. forge) correctly !!!
It treats all dimensions with equal weight, whereas LeCun computes the mean and cov mat of several genuine images and then computes p based on a Gaussian assumption.
Yeah, SigNet just finds a threshold distance and uses that... (which naively weights all dimensions equally)

compute mean and std\_dev of the distance between a matching and non-matching pair on a bunch of training points.

I think that what I should implement (also) is compute the mean and covariance (matices) of the genuine and forge data latent vectors over a bunch of training pairs. (To compute variance, I might need to store the latent vectors in CPU RAM and free the GPU RAM to avoid GPU OOM.)
I can also compute the confusion matrix using the simple threshold distance and my thing and discuss the performance differences.

Maybe I should also try what LeCun did with computing normal distribution for an individual and then using inverse FGSM (aka gradient descent on the input... could I use Pytorch optimizer and stuff for choosing a step value??? no!, don't bother)

If I transfer the latent vectors to the CPU (and don't include the gradients!), then I can easily store the latent vectors for all of the training images!
(should be 128 floats for each, but FaceNet paper says we can compress to 128 bytes ``without loss of accuracy'' ... but they don't say how exactly...)
Then, I could compute the covariances of the clusters and also do some analytics on understanding the clusters.
Yeah, memoizing the latent vectors when doing validation (and finding a threshold(s)) is a really good idea!

\subsubsection{FGSM Attack}
``Here out epsilon of .007 corresponds to the magnitude of the
smallest bit of an 8 bit image encoding after GoogLeNet's conversion to real numbers.''\cite{goodfellow}
[Basically, I'll use the same epsilon value...]

\subsubsection{Latent Vector Size}
From Goodfellow's work, I think it might make sense to try playing with latent vector size...?
Trained on 64, 128, and 256.
The model sizes are: \_, \_, and \_.
Training was way faster for the smaller models of course, but suprisingly so.
1000 per hour vs. \_ vs. \_...

Training loss suggests that none of the models overfit.

smaller models more accuracate after one epoch, but also have many fewer parameters, so this could just mean that they are very underfit (we only did one epoch because training is so slow).
Training is very slow for SigNet256.
(I should introduce the names "SigNet64,128,256" earlier.)
[I should also stick all 3 models in one notebook and attach a name to the models to compute the save\_file str.]

I will use the same epsilon value for all the tests because Goodfellow says that it is the direction that matters and I want them to be apples to apples.

All of the confusion matrices can be seen in section {sec:results}.

% maybe talk about playing with noise too if I actually get around to doing that properly...


\section{just notes lol}
I want to compute my metric of thresholds and prove hopefully prove that results in better accuracy than just optimizing the threshold (by trying all distances) on the training data.
I want to compute a confusion matrix for the 3 models and then recompute it after applying pertebations.

If I have time, I'll also train SigNet128 until it has similar validation loss to that of SigNet64 and then recompute the confusion matrices (and confusion matrices after pertebation).

Hey, does DeepFool do what I'm talking about?
    I should give it another glance before implementing...

With any luck, I'll have results that say which latent vector size is best for accuracy and robustenss to FGSM-like attacks.
!!! Crap!, I think Goodfellow was talking about the input data being high dimensional, not the output.. !!!
    \^check this...
