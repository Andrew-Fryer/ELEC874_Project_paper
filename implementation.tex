\section{Implementation}\label{sec:implementation}

For implementing the SigNet, 3 resources were used.
The SigNet paper was consulted, which describes the model completely\cite{sig_net}.
The Keras code presumably by one of the paper authors was used for clarification\cite{GitHub_sounakdey}.
Lastly, an existing PyTorch implementation of SigNet was used as a starting place\cite{GitHub_signet_pytorch}.

The existing PyTorch implementation first restructured into a Jupyter Notebook.
Then, it was checked against the specification of the network in the paper.
The existing training process did not implement data normalization, so this was implemented as per the paper.
The mean and standard deviation of the images in the training dataset was computed (after spliting the images in the CEDAR dataset into training and validation images and pairing images).
For efficiency, all of the images in the dataset were then resized, transformed (inverted), and normalized according the paper's description and then saved back into png files.

Unfortunately, it was later discovered that the last layer of the network was incorrect in the existing implementation and this went unnoticed.
The Signet implementation in this paper has [last layer output size is 256 instead of 128].
% would training go faster if this was fixed?

The process of preparing the data is all the same.
While the paper uses several datasets, this paper only uses the CEDAR dataset due to time constraints [ref].
% todo: ref

Another small disrepency is in the initialization of model weights.
% todo
% I'm not sure if the biases were correctly initialized to 0 or if the other weights were initialized with the same strategy...

Training was done on Google Colab (using the \_ level tier thing).
The training process took approximately 1 hour per 1000 [cases/datapoints].
The training process was changed to use a dataloader that stores all of the image data in computer memory instead of repeatedly accessing the information from disk in the form of PNG files.
This is feasible because the number of images is far fewer than the number of pairs of images (which are used as [cases/datapoints]).
Unfortunately, this did not seem to speed up the training process significantly.

The model was trained for approximately \_ hours and then evaluated, yeilding an accuracy of 72\%.
Then, it was trained until the end of the first epoch (another \_ hours) and evaluated [validated?] again.

Although the paper trains the model for 20 epochs, training was stopped at 1 epoch due to time constraints.
(Unfortuantely weights from a trained SigNet model could not be obtained.)
