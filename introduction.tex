\section{Introduction}\label{sec:introduction}

Authentication is the process of verifying an individual's identity and is critical to the security of many computer applications and other systems.
Authentication is generally done by testing a user's knowledge, ownership, or inherence factors, which could be a password, a token, or a biometric identifier, respectively\cite{authentication}.

Face recognition and verification is used for authentication in some modern software systems.
With emergence of machine learning, there has been considerable work done on the problem of performing automated authentication based on face verification\cite{sig_net}.

Handwritten signatures have a long history of being a means of authentication\cite{handwriting_survey}.
Therefore, it would be useful to apply techniques used in face verification to signature verification.

A related problem, handwriting recognition, is about determining the meaning of handwriting\cite{handwriting_survey}.
It is complex because ``handwriting is a skill that is personal to individuals'', so the content of the handwriting must be separated from the information about the writer's style and other factors of the environment\cite{handwriting_survey}.
Signature verification involves extracting instead the information about the signer.
It is a very complex problem in biometrics because it requires modelling minute but critical details between a genuine signature and a skilled falsification\cite{sig_net}.

The difficulty of this problem is increased for systems that need to determine whether a signature is a forge with a small number of the individual's genuine signatures because machine learning techniques generally require large quantities of data\cite{handwriting_survey}.

Additionally, if the data used for verification is collected electronically, an attacker could modify it using software with the intent of fooling the verification system.

It has been proved that deep learning models are ``overly confident at points that do not occur in the data distribution, and these confident predictions are often highly incorrect''\cite{goodfellow}, which presents opportunities for a clever attacker to trick a verification system built on many machine learning models including neural networks.
