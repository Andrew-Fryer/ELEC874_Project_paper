% This version of CVPR template is provided by Ming-Ming Cheng.
% Please leave an issue if you found a bug:
% https://github.com/MCG-NKU/CVPR_Template.

\documentclass[final]{cvpr}
%\documentclass[final]{cvpr}

\usepackage{times}
\usepackage{epsfig}
\usepackage{graphicx}
\usepackage{amsmath}
\usepackage{amssymb}

% Include other packages here, before hyperref.

% If you comment hyperref and then uncomment it, you should delete
% egpaper.aux before re-running latex.  (Or just hit 'q' on the first latex
% run, let it finish, and you should be clear).
\usepackage[pagebackref=true,breaklinks=true,colorlinks,bookmarks=false]{hyperref}


\def\cvprPaperID{****} % *** Enter the CVPR Paper ID here
\def\confYear{CVPR 2021}
%\setcounter{page}{4321} % For final version only


\begin{document}

%%%%%%%%% TITLE
\title{Forging Handwritten Signatures}

\author{Andrew Fryer\\
Queen's University\\
99 University Ave, Kingston, ON, Canada\\
{\tt\small andrew.fryer@queensu.ca}
% For a paper whose authors are all at the same institution,
% omit the following lines up until the closing ``}''.
% Additional authors and addresses can be added with ``\and'',
% just like the second author.
% To save space, use either the email address or home page, not both
}

\maketitle


%%%%%%%%% ABSTRACT
\begin{abstract}
% -verification is...
% -facenet is...
% -Signet is...
% -we want to forge...
% -FGSM is...
% -we got good/bad results using _ training set up
   TODO: write last
\end{abstract}

%%%%%%%%% BODY TEXT
\section{Introduction}\label{sec:introduction}
Handwritten signatures have been used for verifying people's identity for a long time.


\section{Related Work}\label{sec:related_work}

\subsection{General Strategy}
Signature verification could be done by training a CNN to output whether or not the input image is of the chosen person.
However, this requires a new network to be trained for each person that the system needs to perform verification for.
Not only is it computationally very expensive to train such a system to perform verification for many people, but also requires many signatures by the person for training.

Instead, a high-level representation of the image can be found that includes only the information needed to distinguish people.
This is effectively dimensionality reduction that preserves only the information that is important for identifying people.
The high-level representation is a vector in what is reffered to as a latent space.
(So, the dimensionality reduction should preserve/extract the structure of the person's face while discarding information about the background and the lighting conditions.)

This idea has a long history for face verification\cite{LeCun}, and has been used for signature verfication much more recently\cite{sig_net}.

\subsection{Handwriting Recognition} % remove if I'm out of space...
``In numerous situations, a pen together with paper or a small notepad is more convenient that a keyboard''\cite{handwriting_survey}.
Early research in automated recognition of handwriting was motiviated by the desire to allow humans to write conveniently and then parse that writing into data inside of a computer.

``On-line'' systems do this given the 2-D coordinates of the writer's pen as a function of time.% and require writing to be captured live by special electronic devices.
``Off-line'' systems only need images of the handwriting, so they are more broadly applicable\cite{handwriting_survey}.

Personal Digital Assistants (PDAs) incorporating on-line systems have been widely used commercially.
These systems have used rule-based, statistical, and implicit methods\cite{handwriting_survey}, espeacially Hidden Markov Models and even convolutional neural networks\cite{389575} (while LeCun's work was in its early stages\cite{mnist}).

None of these methods have been accurate enough to be used commercially on cursive writing\cite{handwriting_survey}.
As of 2000, on-line handwriting recognition was more accurate than off-line recognition because the chronological information is useful\cite{handwriting_survey}.
% However, off-line handwriting recognition is much more broadly applicable because it does not require a special device to capture the handwriting.
% Only a camera is needed to capture an image of the writing.
There have been several attempts made to recreate the temporal data from an image, but these have not been very successful\cite{handwriting_survey}.

% A bank might require an error of 1/ 100,000. ``Current systems are sill several orders of magnitude away''.

\subsection{Contrastive Loss}
``Our approach is to build a trainable system that nonlinearly maps the raw images of faces to points in a low dimensional space so that the distance between these points is
small if the images belong to the same person and large otherwise. Learning the similarity metric is realized by training a network that consists of two identical convolutional
networks that share the same set of weights - a Siamese Architecture''
\begin{figure}[h]
    \begin{center}
        \includegraphics[width=0.8\linewidth]{siamese_architecture.png}
    \end{center}
    \caption{Siamese Architecture. (LeCun 2005)}
    \label{fig:siamese}
\end{figure}
I think LeCun proposed Contrastive Loss...\cite{LeCun}.
[Note: the partitioning and pairing of images is the same idea as SigNet.]
They did this on images of faces.

``A model is constructed of each subject by calculating the mean feature
vector and the variance-covariance matrix using the feature
vectors generated from the first five images of each subject.
The likelihood that a test image is genuine, $p_genuine$, is
found by evaluating the normal density of the test image on
the model of the concerned subject. The likelihood of a test
image being an impostor, $p_imposter$, is assumed to be a constant whose value is estimated by calculating the average $p_genuine$ value of all the impostor images of the concerned
subject. The probability that the given image is genuine is
given by P = $p_genuine$ / ($p_genuine$ + $p_imposter$)''
...so basically they assume that subjects/signers occupy a normal region in the latent space.

\subsection{Triplet Loss}
this paper (where FaceNet got triplet loss...) allows datapoints with the same label to live in multiple clusters:
https://proceedings.neurips.cc/paper/2005/file/a7f592cef8b130a6967a90617db5681b-Paper.pdf
Does FaceNet allow a subject/person to live in several clusters?
The FaceNet paper suggests that each person lives in a single cluster (they imply that and their verifyication strategy is just a distance threshold...)

\subsection{FaceNet}
``FaceNet directly trains
its output to be a compact 128-D embedding using a triplet-based loss function based on LMNN [19]. Our triplets consist of two matching face thumbnails and a non-matching
face thumbnail and the loss aims to separate the positive pair
from the negative by a distance margin. The thumbnails are
tight crops of the face area, no 2D or 3D alignment, other
than scale and translation is performed.''
You could train a classifier and then take an intermediate layer to be a latent space, but this doesn't perform as well...
PCA can improve this, but an end-to-end network performs better.
\cite{face_net}

``The networks are trained by using a combination of classification and verification loss. The verification
loss is similar to the triplet loss we employ [12, 19], in that it
minimizes the L2-distance between faces of the same identity and enforces a margin between the distance of faces of
different identities. The main difference is that only pairs of
images are compared, whereas the triplet loss encourages a
relative distance constraint.''
``Although we did not directly compare to other losses,
e.g. the one using pairs of positives and negatives, as used
in [14] Eq. (2), we believe that the triplet loss is more suitable for face verification. The motivation is that the loss
from [14] encourages all faces of one identity to be projected onto a single point in the embedding space. The
triplet loss, however, tries to enforce a margin between each
pair of faces from one person to all other faces. This allows the faces for one identity to live on a manifold, while
still enforcing the distance and thus discriminability to other
identities.''
When training, they don't use true negatives to adjust weights because the network already knows they are different and so the direction of the gradient is somewhat arbitrary...
[Maybe I should try that...]
They just use a cut off threshold for computing accuracy.
I couldn't find how they chose the threshold...
Part of the results of their paper is that they conclude that using 256-bit over 128-bit latent vectos doesn't really improve accuracy.

\subsection{SigNet}
``Unlike conventional approaches that assign
binary similarity labels to pairs, Siamese network aims to bring
the output feature vectors closer for input pairs that are labelled
as similar, and push the feature vectors away if the input pairs
are dissimilar.''
Uses Contrastive Loss (from LeCun).
Uses 128-bit latent vectors inspired by FaceNet.

``We initialize the weights of the model according to the work
of Glorot and Bengio [20], and the biases equal to 0. We trained
the model using RMSprop for 20 epochs, using momentum rate
equal to 0.9, and mini batch size equal to 128.''

In the paper, SigNet achieves 100\% accuracy on the CEDAR dataset, which contains 
In the validation process, SigNet takes an image and outputs a latent vector.
First, a genuine signature is sent through the network.
Then, either another genuine or a forge is sent throught the network.
The resulting latent vectors are compared using Euclidean distance.
[have I described partitioning yet?]
After computing the latent vector distances for all image pairs in the validation set, the threshold that results in the best accuracy is picked.
This results in 100\% accuracy on the CEDAR dataset.

SigNet was largely inspired by the success of FaceNet.
I haven't seen the SigNet paper discuss why they didn't use triplet loss...

Yo, why does SigNet use contrastive loss instead of triplet loss?
    My guess is that Contrastive Loss is just easier to implement...
\cite{sig_net}
\cite{GitHub_sounakdey}
...and in python: \cite{GitHub_signet_pytorch}

``In this paper, we model an offline writer independent
signature verification task with a convolutional Siamese network.''\cite{sig_net}

\subsection{DeepFool}
\subsection{DEFENSE-GAN}

\subsection{FGSM}
Ian Goodfellow (famous for his development of GANs) proves in this paper (https://arxiv.org/pdf/1412.6572.pdf) that even linear models are susceptable to avasarial attacks consisting of minute pertubations if their dimensionality is large enough.

``Here out epsilon of .007 corresponds to the magnitude of the
smallest bit of an 8 bit image encoding after GoogLeNet's conversion to real numbers.''
[Basically, I'll use the same epsilon value...]
They also should that training on adversarial inputs improves the model's accuracy on adversarial inputs (making it less sussecptible to the FGSM attack).

`` However, noise
with zero mean and zero covariance is very inefficient at preventing adversarial examples. The
expected dot product between any reference vector and such a noise vector is zero. This means that
in many cases the noise will have essentially no effect rather than yielding a more difficult input.''
...this means that adding my noise shouldn't really fool SigNet.

They also discuss ``whether it is better to perturb the input or the hidden layers or both''

Not related but interesting: don't perturb the last hidden layer because there arne't any hidden layers in between it and the output layer to fix the pertebatioons (using  universal approximator theorem)

``An intriguing aspect of adversarial examples is that an example generated for one model is often
misclassified by other models, even when they have different architecures or were trained on disjoint training sets.''


\section{Implementation}\label{sec:implementation}

For implementing the SigNet, 3 resources were used.
The SigNet paper was consulted, which describes the model completely\cite{sig_net}.
The Keras code presumably by one of the paper authors was used for clarification\cite{GitHub_sounakdey}.
Lastly, an existing PyTorch implementation of SigNet was used as a starting place\cite{GitHub_signet_pytorch}.

The existing PyTorch implementation first restructured into a Jupyter Notebook.
Then, it was checked against the specification of the network in the paper.
The existing training process did not implement data normalization, so this was implemented as per the paper.
The mean and standard deviation of the images in the training dataset was computed (after spliting the images in the CEDAR dataset into training and validation images and pairing images).
For efficiency, all of the images in the dataset were then resized, transformed (inverted), and normalized according the paper's description and then saved back into png files.

Unfortunately, it was later discovered that the last layer of the network was incorrect in the existing implementation and this went unnoticed.
The Signet implementation in this paper has [last layer output size is 256 instead of 128].
% would training go faster if this was fixed?

The process of preparing the data is all the same.
While the paper uses several datasets, this paper only uses the CEDAR dataset due to time constraints [ref].
% todo: ref

Another small disrepency is in the initialization of model weights.
% todo
% I'm not sure if the biases were correctly initialized to 0 or if the other weights were initialized with the same strategy...

Training was done on Google Colab (using the \_ level tier thing).
The training process took approximately 1 hour per 1000 [cases/datapoints].
The training process was changed to use a dataloader that stores all of the image data in computer memory instead of repeatedly accessing the information from disk in the form of PNG files.
This is feasible because the number of images is far fewer than the number of pairs of images (which are used as [cases/datapoints]).
Unfortunately, this did not seem to speed up the training process significantly.

The model was trained for approximately \_ hours and then evaluated, yeilding an accuracy of 72\%.
Then, it was trained until the end of the first epoch (another \_ hours) and evaluated [validated?] again.
This time the accuracy was 94\%.

Although the paper trains the model for 20 epochs, training was stopped at 1 epoch due to time constraints and because 94\% accuracy seemed sufficient for the puposes of this paper.
(Unfortuantely weights from a trained SigNet model could not be obtained.)


\section{Results}\label{sec:results}

Here's a table of the accuracy (true/false positive/negative matix).
% todo: compute matrix...

Here's a genuine and a forge and we apply various epsilon values and also repeat...

Here's how well we trick SigNet...

Here's the same thing with 2 genuines of different signatures.

Here's the same thing with 2 genuines of the same signature.


\section{Conclusion}\label{sec:conclusion}

We are/are not able to fool Signet.
This was/was not anticipated.

This qualitatively shows that SigNet's [desicion] is largely influenced by [].


I speculate that SigNet is learning info about the cameras/lighting conditions used for each person/or set up maybe?
...my thinking is that the pertebation looked like background noise, the genuine had all zeros in the background, and the forge had non-zero small values in the background...


\section{Future Work}\label{sec:future_work}

The most surprising conclusion in this paper is that the CEDAR dataset is not useful for evaluating a signature verification system (see section \ref{sec:cedar_flaw}).
The experiments in this paper could be re-conducted on another dataset to properly evaluate the accuracy of SigNet variations.
It may also be possible to transform the CEDAR dataset so that it is useful for our purposes.
One approach is to set all pixels values to 0 or 255 using a threshold.
This would eliminate much of the background noise that is present in the genuine signature images.
However, this may also affect the smoothness of the edges of penstrokes, still enabling the model to learn to distinguish genuine signatures from forgeries without understanding human signatures.

One primary objective of the experiments conducted in this paper was to determine how the latent vector size affects accuracy and susceptibility to attacks involving perturbations.
To create more conclusive data, tables \ref{table:1} and \ref{table:2} should be recreated using random perturbations and using genuine signatures and noise in place of forged signatures.
As stated in section \ref{sec:conclusion}, the accuracies on normal data are not very statistically significant.
The models should be re-initialized and trained many times to produce enough accuracies to be sure of the conclusions made about latent vector size.

The FaceNet paper reports impressive accuracy improvements using the triplet loss function.
It would interesting to see how this would impact the quality of the latent space mapping for SigNet.

Work has already been done in improving network's by training them an adversarial examples created using FGSM and similar methods.

There was one area in particular that is interesting, but couldn't be implemented in the alloted time.
\cite{LeCun} assumes that the cluster of the images of one person's face in the latent space is Gaussian.
We can easily compute the probability that a Gaussian distribution produces a point.
Let's call this probability p\_genuine.
However, computing the probability that a point belongs to the impostor class is tricky.
In the paper, they average the p\_geuine probabilities for the impostor images for a subject to produce p\_impostor.
p\_impostor is used to normalize the p\_genuine value of an image.
Then a threshold is used on the resulting pseudo-probability.
(Note that with a threshold of 0.5, something like half of the impostor points would be classified as being genuine when they are completely disjoint.)

They compute a different distribution and threshold for each person using their first 5 images.
To be consistent with the SigNet paper, this paper uses a one-shot approach, so only one genuine signature is given alongside the signature in question.
Therefore, we cannot compute a distribution for each person.
Instead, we could compute a distribution for all genuine signatures.

Their approach makes sense, but it may be advantageous to eliminate the need to pick a threshold.
An alternate way to compute the probability that a point belongs to the impostor class is to make a somewhat crude assumption that the impostor class is also Gaussian.

The formula for deciding whether or not a point $x$ (difference between latent vectors) should be classified as genuine is given and reduced in figure \ref{proof:gaussian}.
The $<$ symbol is the computer science ``less than'' comparison operator rather than denoting an inequality, and $g$ is for ``genuine'' and $i$ is for ``impostor''.

\begin{figure}
\begin{align*}
is\_genuine & = P_g > P_i&\\
        & = \Pr(c_g|x) > \Pr(c_i|x)&\\
        & = \frac{\Pr(x|c_g)\Pr(c_g)}{Pr(x)} > \frac{\Pr(x|c_i)\Pr(c_i)}{Pr(x)}&\\
        & = \Pr(x|c_g)\Pr(c_g) > \Pr(x|c_i)\Pr(c_i)&\\
        & assume\ that\ \Pr(c_g) = \Pr(c_i)&\\ % (could be tuned for different datasets)
        & = \Pr(x|c_g) > \Pr(x|c_i)&\\
        & = \mathcal{N}(\mu_g,\,\Sigma_g^{2}) > \mathcal{N}(\mu_i,\,\Sigma_i^{2})&\\
        & = \frac{1}{\sqrt{(2\pi)^{n}|\Sigma_g|}}exp(-\frac{1}{2}(x-\mu_g)'\Sigma_g^{-1}(x-\mu_g))&\\
        & > \frac{1}{\sqrt{(2\pi)^{n}|\Sigma_i|}}exp(-\frac{1}{2}(x-\mu_i)'\Sigma_i^{-1}(x-\mu_i))&\\
        & = |\Sigma_g|^{-1/2}exp(-\frac{1}{2}(x-\mu_g)'\Sigma_g^{-1}(x-\mu_g))&\\
        & > |\Sigma_i|^{-1/2}exp(-\frac{1}{2}(x-\mu_i)'\Sigma_i^{-1}(x-\mu_i))&\\
        & assume\ \mu_g = \mu_i = 0\ and\ take\ ln&\\
        % & = take ln&\\
        % & = simplify&\\
        & = -x^T * \Sigma_g^-1 * x > -x^T * \Sigma_i^-1 * x &\\
        &    + 2 * (ln(|\Sigma_i|^{-1/2}) - ln(|\Sigma_g|^{-1/2}))&\\
\end{align*}
\caption{Reduction of Gaussian Probability Comparison for a Smart Threshold Function}
\label{proof:gaussian}
\end{figure}

The assumption the $\Pr(c_g) = \Pr(c_i)$ could be revoked and replaced by a constant ratio of the probabilities and tuned for the system based on the class priors and the desired confidence interval that a pair classified as genuine is in fact genuine.

This threshold strategy can be implemented efficiently by memoizing the term $2 * (ln(|\Sigma_i|^{-1/2}) - ln(|\Sigma_g|^{-1/2}))$ and by memoizing the latent vectors for all images in the dataset.
The latter should result in a large performance improvement because there are many more pairs of images than images in the validation set.


{\small
\bibliographystyle{ieee_fullname}
\bibliography{egbib}
}

\end{document}
